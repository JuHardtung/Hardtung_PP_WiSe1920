%!TEX root = ../Hardtung_PP_WiSe1920.tex

\section{Introduction}
\label{sec:introduction}

Creating instructions for folding \gls{origami} models is a tedious and time consuming task. These so called origami diagrams (see Fig \ref{fig:diagramExample}) have to be accurate representations of the paper for every folding step, in order to unambiguously explain how to fold the model. Every flap, crease and edge has to be drawn (see Section \ref{generalRules} for exceptions), which makes the process slow and especially error-prone for complex models.
So far, diagrams got either drawn by hand, or created with the help of digital vector programs like Inkscape or Adobe Illustrator. Even though the digital diagramming helped with the accuracy of the finished instructions, the task itself was still quite time consuming.\cite{???}

The goal for this project is to develop a desktop application, that implements features specifically for the origami diagramming process. The standardized symbols and overall notations have to be included and the program has to offer functions that increase the efficiency of creating diagrams.