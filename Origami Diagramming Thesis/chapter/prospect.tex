%!TEX root = ../Hardtung_PP_WiSe1920.tex

\section{Prospect and further possible Features}
\label{sec:prospect}

Despite the mostly finished state of \gls{origrammer} there are various additional features that could prove useful once implemented. This section will list and explain missing features and general usability improvements and the following subsections are roughly sorted by descending importance. Though this should not be seen as an extensive list of missing essential features, but rather as a prospect on what the further development of this diagramming program could look like. 

\subsection{Export to .pdf}

Exporting a finished diagram to the Portable Document Format (PDF) would provide an easy way to share diagrams created within the Origrammer. This would provide greater accessibility to the diagrams while also removing the requirement of using the Origrammer just for displaying purposes. This would be especially helpful for folders who simply want to follow the diagrams.

\subsection{``Intelligent'' Input functions}

So far every line, arrow or symbol has to be manually placed by hand. In order to increase productivity some ``intelligent'' functions could be implemented. These functions could for example place, rotate and scale arrows automatically after placing a valley/ mountain fold on the paper.

\subsection{Input Validation}

As already mentioned in Section \ref{sec:breakingCalc}, a proper input validation is currently missing in a couple of places. 

\subsection{Further polishing}



\subsection{General usability improvements}

Improving the usability of Origrammer can be achieved by a multitude of measures.
Simple tooltips or short explanatory texts would remove some uncertainty from novices as well as giving reminders for experienced users. When taking this further, a help section

Using current Human Computer Interaction (HCI) methods to increase usability would also be a logical step in the future. This process can be further improved by conducting external usability tests or creating surveys