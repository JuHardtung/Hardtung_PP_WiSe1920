%!TEX root = ../Hardtung_PP_WiSe1920.tex

\section{Requirement Analysis}
\label{sec:requirementAnalysis}

After collecting and categorizing all diagramming notations, the actual requirements of the planned system can be defined more precisely. On basis of the previous section (Section \ref{sec:conventions}) three different requirement fields become apparent.

\begin{itemize}
	\item Diagramming Symbols
	\item General Diagramming Rules
	\item General Program Tools
\end{itemize}

\subsection{Diagramming Symbols}

Beginning with the most important field, the diagramming symbols form the basis of diagramming and implementing all of them is imperative. Without all the different lines, arrows and other symbols there will be ambiguity in the later folding instructions.

\subsection{General Diagramming Rules}

This field contains more abstract rules and best practices while creating origami diagrams. Including them in the system has to be decided on a case to case basis, as some rules have to be kept in mind by the artist during diagramming. Things like using the right \emph{origami grammar} or \emph{distorting the model in order to show otherwise hidden layers and edges} (see Section \ref{sec:generalRules}) rely heavily on the experience of the artist. 

Though in the future some of these rules could be displayed as a hint within the program. And in the case of the right origami grammar, it could be thinkable to develop a system that automatically adds the description text of a step based on the symbols used.

\subsection{General Program Tools}

Outside of providing all the diagramming symbols and including the general diagramming rules where possible, the program has to offer common tools that have become standard in most vector programs. 

\begin{itemize}
	\item \textbf{Select, Move, Resize, Rotate:} editing an object (line, symbol etc.)
	\item \textbf{Delete:} delete diagramming symbol or whole step
	\item \textbf{Duplicate:} duplicate diagramming symbols or whole step
	\item \textbf{Copy \& Paste:} copy or past an object (line, symbol etc.)
	\item \textbf{Undo \& Redo:} undo or redo the last action (adding a line, symbol etc.)
	\item \textbf{Zoom, Rotate, Flip:} editing a step
	\item \textbf{Save, Open:} saving and opening an origami diagram
	\item \textbf{Export:} exporting a finished diagram to e.g .pdf 
\end{itemize}

Even though these features are quite important for the later usability of Origrammer, the focus was set on implementing all origami specific tools first, as this provides the biggest unique feature set when comparing to already existing vector programs.