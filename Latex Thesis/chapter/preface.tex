%!TEX root = ../VorlageBA.tex
\chapter*{Vorwort (optional)}
\addcontentsline{toc}{chapter}{Vorwort}
Das Vorwort ist optional.\\ \\

% \noindent{}
\fcolorbox{black}{sourcegray}{\parbox{\textwidth}{
Sollte es formale Vorgaben seitens Betreuer oder des Institutes zur formalen Gestaltung Ihrer wissenschaftlichen Arbeit geben, verwenden Sie diese. Existieren keine Vorgaben oder decken diese nicht alle formalen Bereiche ab, können Sie diese Vorschläge als Hilfestellung verwenden und klären Sie diese mit dem Betreuer ab.}}

\vspace{0.8cm}

\noindent{}\fcolorbox{black}{sourcegray}{\parbox{\textwidth}{Diese Vorlage orientiert sich an den allgemein anerkannten formalen Richtlinien (Seitenränder, Schriftgröße, etc.) für Bachelor-, Diplom- und Masterarbeiten, sowie an den Vorgaben des Prüfungsausschusses des Fakultät für Informatik und Ingenieurwissenschaften der Technische Hochschule Köln. Die Verzeichnisse (Abbildungs-, Tabellen-, Abkürzungsverzeichnis) wurden in dieser Vorlage jedoch an das Ende gestellt. Der Prüfungsausschuss sieht diese gerne direkt nach dem Inhaltsverzeichnis, daher sollten Sie dies mit Ihrem jeweiligen Betreuer abklären.}}